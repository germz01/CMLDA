\chapter{Optimizers} % (fold)
\label{cha:optimizers}
	\section{Stochastic Gradient Descent} % (fold)
	\label{sec:sgd}
	
	% section struttura_della_rete (end)

		\subsection{Momentum} % (fold)
		\label{sec:momentum}
		
		% section forward_e_back_propagation (end)

		\subsection{Regolarizzazione} % (fold)
		\label{sec:regolarizzazione}

	\section{Nonlinear Conjugate Gradient} % (fold)
	\label{sec:nonlinear_conjugate_gradient}
	
		An intresting optimization ables to lead to an improvement of the performances of the Neural Network, is the use of high-order information during the training phase.

		We can, indeed, approximate the \textit{loss function} in a quadratic form, around a given point \textbf{W}, using the Taylor approximation:

		 \begin{equation} 
		 	\label{quadratic}
		    \mathcal{E}(\textbf{W} + \Delta\textbf{W}) = \mathcal{E}(\textbf{W})+\textbf{g}\mathcal{E}^T\Delta\textbf{W}+ \frac{1}{2}\Delta\textbf{W}^T\textbf{H}\Delta\textbf{W},
		 \end{equation} 

		where \textbf{H} is the \textit{Hessian} matrix and \textbf{g} the gradient vector, getting the benefit of choosing the search direction and the step size more carefully by using information from the second order approximation.

		Given Eq. \ref{quadratic}, the optimum adjustment to apply to the weights of the network should be 
 		\begin{equation} 
		 	\label{weight_star}
			\Delta\textbf{W}^* = \textbf{H}^{-1}\textbf{g},
		\end{equation} 

		carrying the weight of the computation of the inverse of the Hessian.

		In order to avoid this expensive computation, we can use the \textit{Conjugate Gradient} methods, which are a class of iterative second-order optimization methods, derived from the steepest-descent algorithm, that ensure low memory requirements.

		In this way, the adjustment to the synaptic weights of the network is computed as:
		 \begin{equation} 
		 	\label{weight}
		    \Delta\textbf{W} = \alpha\textbf{d},
		 \end{equation} 
		where $\alpha$ is the learning rate and \textbf{d} is the new direction found.

		In our case, the nonlinear conjugate gradient methods are designed to solve the following minimization problem:

		\begin{mini} 
		  {\textbf{W}\in \mathbb{R}^n}{\mathcal{E}(\textbf{W}),}{}{}
		\end{mini}

		where $\mathcal{E}$ is the loss function and \textbf{W} are the synaptic weights of the network. 

		As showed in the pseudocode \ref{cgd}, the iterative formula generates a sequence of weights $\{W_k\}$, for every epoch of training \textit{k}, as:

		\begin{equation} 
			\textbf{W}_{k+1} = \textbf{W}_{k} + \alpha\textbf{d}_k, \text{  }\text{  }\text{  } \textit{k} = 0,1,...,
		\end{equation}

		where $\alpha_k$ is a learning rate and $\textbf{d}_k$ is a descent direction. These are the new synaptic weights computed with the adjustment of eq.\ref{weight}.

		% Because the error function E(~,) is nonquadratic, the algorithm will not necessarily con- verge in N steps. If the algorithm has not converged after N steps, the algorithm is restarted, i.e., initializing fik+J to the current steepest descent direction ~k+~  A Scaled Conjugate Gradient Algorithm for Fast Supervised Learning p4


		\subsection{Search Direction}
		\label{sub:search_direction}
			The direction $\textbf{d}_k$ holds the sequent property:
			\begin{equation} 
			\textbf{d}_k^T\textbf{H}\textbf{d}_{tk-1} = 0,
			\end{equation}

		 	that means it is conjugate to the previous direction $\textbf{d}_{k-1}$. Furthermore, it doesn't need to know all the 	previous directions, but it only needs the last one, which is why it requires very little storage and computation.

			When dealing with quadratic functions, this method keeps the progress obtained so far in the minimization of the loss function, by ensuring that the gradient along the previous direction does not increase.
			Anyway, it's worth to underline that this method can also be applyed with nonlinear functions: in this case, it should be necessary to restart the process, since there is no assumption that the conjugate directions previously found are still at the minimum of the function. 

			Each new direction it's a linear combination of the steepest descent -\textbf{g} and the previous direction $\textbf{d}_{k-1}$, and it is defined as:
			\begin{equation}
			\label{dir}
			  \textbf{d}_k=\begin{cases}
			    -\textbf{g}_0, & \text{if $k=0$};\\
			    -\textbf{g}_k + \beta_k\textbf{d}_{k-1}, & \text{otherwise,}
			  \end{cases}
			\end{equation}

			where $\beta_k$ is a scalar, to be determined, that says how much of the previous direction should be added to the newest one. When applied to minimize a strictly convex quadratic function, it ensure that the directions $\textbf{d}_{k}$ and $\textbf{d}_{k-1}$ are conjugate with respective to the Hessian of the objective function, that is the property \ref{sub:search_direction} holds. 

			Of course, the first search direction when $k = 0$ is defined as the steepest descent direction at the initial weight $\textbf{W}_0$ while, for $k > 1$, a minimization along each of the search direction is performed.

			Since it may be that the direction found is not a descent direction of the objective function, another modified search direction, proposed by Zang et al.\ref{zhang}, has been tested in the project. It ensures sufficient descent indipendent of the line search used or the convexity of the objective function and is defined as follows:

			\begin{equation}
				\textbf{d}_k^+=\begin{cases}
			    -\textbf{g}_0, & \text{if $k=0$};\\
			    -(1 + \beta_k\frac{\textbf{g}_k^T\textbf{d}_{k}}{\|\textbf{g}_k\|})\textbf{g}_k + \beta_k\textbf{d}_{k-1}^+, & \text{otherwise.}
			  \end{cases}
			\end{equation}



		\subsection{Beta}
		\label{sub:beta}
			What really makes the difference in the computation of the conjugate gradient algorithm, is the choice of the method used to compute the $\beta$ coefficient.

			Infact, there has been proposed various choices for computing it, each one giving different efficiency and properties.

			The formulas tested in our implementation are four: the Fletcher-Reeves (FR), the Polak-Ribierère (PR), the Hestenes-Stiefel (HS) and a Modified Hestenes-Stiefel ($MHS^+$).

			One of the properties that must be garanteed, is the global convergence of the method. Since, in our network, we are dealing with a nonquadratic loss function, the direction computed as in eq. \ref{dir} could not be a descent direction. In order to avoid this issue, all the methods have been modified as follows, ensuring the global convergence:
			\begin{equation}
			\label{beta_max}
				 \beta^+ = max\{\beta, 0\}.
			\end{equation}

			This change provides a sort of restart of the algorithm, in case the $\beta$ found is negative. This is equivalent to forget the last search direction and start again the search from the steepest descent direction. Furthermore, because of the nonquadratic nature of the error function, the algorithm will not necessarily converge in N steps, as it usually does when applied to quadratic functions. The use of $\beta$ in eq. \ref{beta_max} is similar to adopt the strategy of restarting the algorithm after N steps, initializing $d_k$ to the current steepest descent direction.% A Scaled Conjugate Gradient Algorithm for Fast Supervised Learning p4

			\begin{equation}
			\label{betas}
				 \beta^{PR}_k = \frac{\textbf{g}_k^T(\textbf{g}_k-\textbf{g}_{k-1})}{\|\textbf{g}_{k-1}\|^2}, \text{ }
 				 \beta^{FR}_k = \frac{\|\textbf{g}_{k}\|^2}{\|\textbf{g}_{k-1}\|^2}, \text{ }
 				 \beta^{HS}_k = \frac{\textbf{g}_k^T(\textbf{g}_k-\textbf{g}_{k-1})}{(\textbf{g}_k-\textbf{g}_{k-1}^T\textbf{d}_{k-1})}.
			\end{equation}

			The HS and the PR methods in eq. \ref{betas} have very similar performances and they are two of the most efficient conjugate gradient methods, but they are not globally convergent for nonlineat function. That's why the modification of eq. \ref{beta_max} has been adopted. 
			Moreover, the HS method is considered superior to other methods when applied to nonquadratic functions.

			For what concernes the FR method (also described in eq.\ref{betas}), it requires a constrain on the parameters of the inexact line search procedure of section \ref{sub:line_search}, used to identify the right step length $\alpha$. In particular, it requires that $\sigma_1 < \sigma_2 < 0.5$ in order to garantee that the Armijo Wolfe conditions are satisfied, and it seems to be less efficient and robust than the other methods.%p121 nonlinear conjugate gradient method cpt5
			Anyway, by imposing this condition, the FR method is globally convergent even when dealing with nonlinear functions.

			The last method tested is the $MHS^+$ (eq. \ref{mhs}, a modified version of the Hestenes-Stiefel one.  %a new conjugate gradient algorithm for training neural network
			It garantees sufficient descent with inexact line search and is based on a modified secant equation which approximates the second order information of the loss function with high order accuracy. Moreover, it is globally convergent.
			
			It is defined as follows:

			\begin{equation}
			\label{mhs}
 				 \beta^{MHS}_k = \frac{\textbf{g}_k^T \~y_{k-1}^*}{\textbf{d}_{k-1}^T\~y_{k-1}^*},
			\end{equation}




		\subsection{Line Search} % (fold)
		\label{sub:line_search}
			Once computed the new direction \textbf{d} involved in the new weights \textbf{W}+$\alpha$\textbf{d}, a line search has to be implemented in order to find the right step size which minimize the loss function.

			The step size $\alpha$ is nothing more than a scalar: the learning rate for the conjugate gradient algorithm, which tells how far is right to move along a given direction. 

			So, fixed the values of the weights \textbf{W} and the descent direction \textbf{d}, the main goal is to find the right value for $\alpha$ that is able to minimize the loss function:

			 \begin{mini} 
			   {\alpha}{\mathcal{ E}(\textbf{W} + \alpha\textbf{d}).}{}{}
			 \end{mini}

			Of course, we have to deal with a tradeoff: we want a good reduction, but we can't spend too much time computing the exact value for the optimum solution. So, the smarter way to get it is to use an inexact line search, that try some candidate step size and accepts the first one satisfying some conditions.  

			This search is performed in two phases: 
			\begin{itemize}
			\item a \textit{bracketing phase}, that finds an initial interval containing a minimizer;
			\item an \textit{interpolation phase} that, given the interval, finds the right step length in it.
			\end{itemize}

			We decided to use one of the most popular line search condition: the \textit{Armijo-Wolfe} condition.

			The search for the better $\alpha$ is led by two condition:
			\begin{itemize}
			\item the \textit{Armijo} one: 
			\begin{equation}
			\mathcal{E}(W_k+\alpha _kd_k)\leq \mathcal{E}(W_k)+\sigma_1\alpha\nabla \mathcal{E}_k^Td_k
			\end{equation}

			which ensure that $\alpha$ gives a sufficient decrease of the objective function, being this reduction proportional to the step length $\alpha$ and the directional derivative $\nabla \mathcal{E}_k^Td_k$.

			The constant $\sigma_1$ has been set $\sigma_1=10^{-4}$, since it is suggested in literature to be quite small.

			\item the \textit{Strong Wolfe} condition:
			\begin{equation}
			|\nabla \mathcal{E}(W_k+\alpha _kd_k)^Td_k|\leq \mathcal{E}(W_k)+\sigma_2|\nabla \mathcal{E}_k^Td_k|
			\end{equation}
			which garantees to choose steps whose size is not too small.

			It is also known as curvature condition and ensures that, moving of a step $\alpha$ along the given direction, the slope of our function if greater than $\sigma_2$ times the original gradient (if the slope is only slightly negative, the function cannot decrease rapidly along that direction, so it's better to stop the search).

			In this case, the constant $\sigma_2$ is equal to 0.1, since a smaller value gives a more accurate line search.
			Futhermore, having choosen the strong condition, which doesn't allow the derivative to be too positive, we are sure that the $\alpha$ found lies close to a stationary point of the function.
			\end{itemize}

			The algorithm satisfing the Strong Wolfe conditions is implemented through three functions, as described in the pseudocodes \ref{linesearch}, \ref{zoom}, \ref{bisection}:  \texttt{line\_search}, \texttt{zoom} and \texttt{interpolate\_alpha}.

			Since two consecutive values may be similar in finite-precision arithmetic, we set a threshold in both the \texttt{line\_search} and  \texttt{interpolate\_alpha} functions, which garantees that the algorithm stops if two values of $\alpha$ are too close or if the maximum number of iterations has been reached.

			The \texttt{line\_search} function try to find and return a good $\alpha$; if it fails, it returns an interval in which continue the searching, invoking the \texttt{zoom} function, which decreases the size of the interval, until it finds and returns a good step length.

			\texttt{Zoom} invokes another function, \texttt{interpolate\_alpha}, which is nothing more than the implementation of a bisection interpolation in order to find a trial $\alpha$ inside the given interval. 



			\begin{algorithm}
			\caption{Line Search}\label{linesearch}
			\begin{algorithmic}[1]
			% \Input {\alpha_1>0, \alpha_{max}}\\
			% \Output {\alpha_*}\\
			\Procedure{line\_search}{}\\
			\State $\alpha_0 \gets \textit{0};$
			\State $i \gets \textit{1};$
			\While {$i \leq max\_iter$} 
			 		\State  $ \text{Evaluate}\mathcal{E}(\alpha_i);$
					\If  { $[\mathcal{E}(\alpha_i) > \mathcal{E}(0) + \sigma_1\alpha_i\nabla\mathcal{E}_0^Td_0] \text{  or  } [\mathcal{E}(\alpha_i)\leq \mathcal{E}(\alpha_{i-1}) \text{  and  } i>1]  $ }
						\State $\alpha_* \gets \textbf{zoom}(\alpha_{i-1},\alpha_{i});$
						\Return $\alpha_*; $
					\EndIf
					\State   $\text{Evaluate } \nabla\mathcal{E}_{i} $
					\If  { $|\nabla\mathcal{E}_{i}| \leq -\sigma_2\nabla\mathcal{E}_0^Td_0 $}
						\State $\alpha_* \gets \alpha_i; $
						\Return $\alpha_*; $
					\EndIf
					\If  { $\nabla\mathcal{E}_{i} \geq 0$ }
						\State $\alpha_* \gets \textbf{zoom}(\alpha_{i},\alpha_{i-1}); $
						\Return $\alpha_*; $
					\EndIf
					\If  { $(|\mathcal{E}_{i} - \mathcal{E}_{i-1}| \leq threshold$ }
						\State $\alpha_* \gets\alpha_{i}$
						\Return $\alpha_*;$
					\EndIf
					\State $\text{Choose  } \alpha_{i+1}  \in (\alpha_i, \alpha_{max});$
					\State $i \gets i+1; $
			\EndWhile
			\EndProcedure
			\end{algorithmic}
			\end{algorithm}



			\begin{algorithm}
			\caption{Zoom}\label{zoom}
			\begin{algorithmic}[1]
			\Procedure{zoom}{}
			\Repeat 
				\State  $ \alpha_j \gets \textit{interpolate\_alpha}(\alpha_{lo}, \alpha_{hi}); $
				\State $ \text{Evaluate }\mathcal{E}(\alpha_j);$
				\If {$[\mathcal{E}(\alpha_j) > \mathcal{E}(0) + \sigma_1\alpha_j\nabla\mathcal{E}_0^Td_0] \text{  or  } [\mathcal{E}(\alpha_j)\leq \mathcal{E}(\alpha_{lo}) $ }
					\State $\alpha_{hi} \gets \alpha_{j};$
				\Else 
					\State $\text{Evaluate } \nabla\mathcal{E}_j^Td_j; $
					\If { $|\nabla\mathcal{E}_j^Td_j| \leq -\sigma_2\nabla\mathcal{E}_0^Td_0$}
						\State $\alpha_* \gets \alpha_j; $
						\Return $\alpha_*; $
					\EndIf
					\If { $\nabla\mathcal{E}_j^Td_j (\alpha_{hi}-\alpha_{lo}) \geq 0 $ }
						\State $\alpha_{hi} \gets \alpha_{lo}; $
					\EndIf
					\If  { $(|\mathcal{E}_{j} - \mathcal{E}_{0}| \leq threshold$ }
						\State $\alpha_* \gets\alpha_{j}$
						\Return $\alpha_*;$
					\EndIf
				\EndIf
				\State $\alpha_{lo} \gets \alpha_{j}; $
			\EndRepeat 
			\EndProcedure
			\end{algorithmic}
			\end{algorithm}


			\begin{algorithm}
			\caption{Interpolate}\label{bisection}
			\begin{algorithmic}[1]
			\Procedure{interpolate\_alpha}{}
			\State $i \gets \textit{1};$
			\While {$i \leq max\_iter$} 
				\State $\alpha_{mid} \gets (\alpha_{hi}-\alpha_{lo})/2$
				\State  $ \text{Evaluate }\mathcal{E}(\alpha_{mid});$
			 	\If  { $[\mathcal{E}(\alpha_{mid}) == 0]  \text{  or  }  [(\alpha_{hi}-\alpha_{lo})/2 < threshold] $ }
						\Return $\alpha_{mid}; $
				\EndIf	
				\State   $\text{Evaluate } \mathcal{E}(\_{mid}alpha_{lo}); $
				\If  { $sign(\mathcal{E}(\alpha_{mid})) == sign(\mathcal{E}(\alpha_{lo}))$ }
					\State $\alpha_{lo} \gets \alpha_{mid}; $
				\Else
					\State $\alpha_{hi} \gets \alpha_{mid}; $
				\EndIf
				\State $i \gets i+1; $
			\EndWhile
			\EndProcedure
			\end{algorithmic}
			\end{algorithm}
